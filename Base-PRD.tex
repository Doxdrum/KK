\documentclass[aps,prd,12pt,superscriptaddress,showpacs,showkeys,reprint]{revtex4-1}

\usepackage{amsmath,amsthm,latexsym,amssymb,amsfonts}
\usepackage{xcolor}
\usepackage[%
  colorlinks=true,
  urlcolor=blue,
  linkcolor=blue,
  citecolor=blue
]{hyperref}
\usepackage{etoolbox}
\usepackage{breqn}

%% \makeatletter
%% \let\cat@comma@active\@empty
%% \makeatother

%------------------
%--------- Definitions
%------------------
\input{Def-article.tex}


%% \hypersetup{%
%%   pdftitle={Einstein's Gravity from an Affine Model},
%%   pdfauthor={Oscar Castillo-Felisola,}{Aureliano Skirzewski},
%%   pdfkeywords={Affine Gravity,} {Torsion,} {Generalised Gravity.},
%%   pdflang={English}
%% }


%------------------
%--------- Document
%------------------
\begin{document}

\title{Kaluza--Klein reduction and symmetry conditions}

\author{Oscar \surname{Castillo-Felisola}}
%% \email{o.castillo.felisola@gmail.com}
\affiliation{\CCTVal.}
\affiliation{\UTFSM.}

\author{Cristobal \surname{Corral}}
\affiliation{\CCTVal.}
\affiliation{\UTFSM.}

\author{Sim\'on \surname{del Pino}}
%% \email{askirz@gmail.com}
\affiliation{\PUCV.}

%% --------- Abstract
\begin{abstract}
  We ...
\end{abstract}

%% \pacs{02.40.Ma,04.50.Kd,04.90.+e}
%% \keywords{Affine Gravity, Torsion, Generalised Gravity.}


\maketitle

\section{\label{intro}Introduction}

The standard theory of gravity, General Relativity, assumes vanishing torsion, and is a theory for the metric field. However, it is well-known that an equivalent formulation can be stated in terms of two fields ---the vielbein and the spin connection--- which in the case of vanishing torsion are related, albeit in general are independent.

\section{Isometries of Kaluza--Klein ansatz}

The Kaluza--Klein program aims to obtain an effective lower dimensional physical theory from a higher dimensional one, by means of what is known as dimensional reduction.

In what follows we shall consider a $D+1$-dimensional gravitational theory without and with torsion) and obtain the reduced theory on a 1-sphere. For our purpose a good $D+1$-metric ansatz will be~\cite{Duff:1986hr,PopeKK}
\begin{equation}
  \de{\hat{s}}^2 = e^{2\alpha\phi} \de{s}^2 + e^{2 \beta \phi} \left( \de{\xi} + \Ag[1] \right)^2,
  \label{KKansatz}
\end{equation}
where the hatted quantities are defined on the whole $D+1$-dimensional spacetime, $\de{s}^2$ is the line element of the codimension one spacetime, $\xi$ is the extra dimension, $\Ag[1]$ and $\phi$ are a vector and scalar field on the $D$-dimensional spacetime. The metric ansatz is equivalent to a vielbein choice
\begin{equation}
  \begin{split}
    \hvif{a} &= e^{\alpha \phi} \vif{a}, \\
    \hvif{*} &= e^{\beta \phi} \left( \de{\xi} + \Ag[1] \right),
  \end{split}
\end{equation}
where the asterisk denotes the index along the extra dimension. These vielbeins yield 
\begin{equation}
  \begin{split}
    \hspife{ab}{} &=\spife{ab}{} + \alpha e^{-\alpha \phi} \left( \pau{b} \phi \, \hvif{a} - \partial^{a} \phi \, \hvif{b} \right) \\
    & \quad - \frac{1}{2} e^{(\beta - 2\alpha) \phi} \Fg^{a b} \, \hvif{*}
    \\
    \hspife{*}{a} &= \beta e^{-\alpha \phi} \pa{a} \phi \, \hvif{*} + \frac{1}{2} e^{(\beta - 2\alpha) \phi} \Fg_{a b} \, \hvif{b} 
  \end{split}
\end{equation}

\subsection*{Killing vectors}

The Killing vectors of the metric ansatz in Eq.~\eqref{KKansatz} are 
\begin{equation}
  \begin{split}
    \Xi^\mu &= \Xi^\mu(x), \\
    \Xi^*  &= c \xi + \zeta(x).
  \end{split}
  \label{Kvector}
\end{equation}
From these Killing vectors, the field transformations are
\begin{align}
  \label{Liephi}
  \pounds_\Xi \phi    &= \Xi^\mu \pa{\mu} \phi + \frac{c}{\beta}, \\
  \label{LieA}
  \pounds_\Xi \Ag_\nu  &= \Xi^\mu \pa{\mu} \Ag_\nu + \Ag_\mu \pa{\nu} \Xi^\mu + \pa{\nu} \zeta(x) - c \Ag_\nu, \\
  \label{Lieg}
  \pounds_\Xi g_{\nu\rho} &= \Xi^\mu \pa{\mu} g_{\nu\rho} + g_{\mu\rho} \pa{\nu} \Xi^\mu + g_{\nu\mu} \pa{\rho} \Xi^\mu - \frac{2\alpha c}{\beta} g_{\nu\rho}.
\end{align}


\subsection*{Invariant spin connection}

\textcolor{red}{In the usual Kaluza--Klein dimensional reduction (see, for example, Ref.~\cite{PopeKK}), there is an induced torsion in the effective theory (compare with the results in Refs.~\cite{PhysRevD.17.3141,PhysRevD.19.430}).}

At this stage, we will find the most general connection compatible with the symmetries generated by the Killing vectors in Eq.~\eqref{Kvector}. Although it is possible to find such a field by solving the Killing equation for a connection, say~\cite{Tilquin:2011bu}
\begin{equation}
  \begin{split}
    \pounds_\Xi \connn{\mu}{\rho}{\nu} &= \Xi^\sigma \pa{\sigma} \connn{\mu}{\rho}{\nu} - \connn{\mu}{\sigma}{\nu} \pa{\sigma} \Xi^\rho + \connn{\sigma}{\rho}{\nu} \pa{\mu} \Xi^\sigma \\
    & \quad + \connn{\mu}{\rho}{\sigma} \pa{\nu} \Xi^\sigma + \frac{ \partial^2 \Xi^\rho }{ \partial x^\mu \partial x^\nu} \\
    &= 0,
  \end{split}
\end{equation}
the decomposition of a general connection into the Levi-Civita connection plus a contorsion tensor, allows us to simplify the task. Therefore, instead of analyzing the symmetries of the contorsion tensor, or equivalently the torsion tensor.

The what follows we shall consider the following spin connection ansatz
\begin{equation}
  \begin{split}
    \hspif{ab}{} &= \spif{ab}{} + \psi^{ab} \hvif{*}, \\
    \hspif{a*}{} &= \bs{\lambda}^a + \chi^a \hvif{*},
  \end{split}
\end{equation}
which yield a torsion 2-form given by
\begin{equation}
  \begin{split}
    \hTf{a} &= e^{\alpha \phi} \big( \Tf{a} + \alpha \df[\phi] \we \vif{a}  - \psi^a{}_b \vif{e} \we \hvif{*} + e^{- \alpha \phi} \bs{\lambda}^a \hvif{*} \big)
  \end{split}
\end{equation}

\subsection*{Transformation under gauge transformations}

\section{Symmetries of the complete spacetime}

\subsection*{Gauge theories of gravity}

\section{Residual symmetries in the codimension one spacetime}

\subsection*{Spontaneous symmetry breaking and dimensional reduction}

\subsection*{Transformation of reduced fields}

\subsection*{New conservation theorems}

\section{Analysis of the minimal model with torsion}

\subsection*{Gau\ss--Bonnet theory in five dimensions}

\section{Discussion and conclusions}

%%%%%%%%% ACKNOWLEGMENTS %%%%%%%%%
\begin{acknowledgments}
  We thank to A. Toloza and ...  for their helpful discussions and inspiring comments, to J. Zanelli for his suggestions on the physical insight into the problem and careful but critical review of the manuscript.
  This work was partially supported by CONICYT (Chile) under project No. 79140040.
\end{acknowledgments}

%%%%%%%%% APPENDIXES %%%%%%%%%
\appendix

\section{Collineation equations for the torsion tensor}

Given the (anti)symmetry of the torsion tensor, the collineation equation is
\begin{equation}
  \begin{split}
    \pounds_\Xi \TORS{\mu}{\rho}{\nu} &= \Xi^{\hat{\sigma}} \PA{\sigma} \TORS{\mu}{\rho}{\nu} + \TORS{\sigma }{\rho}{\nu} \PA{\mu} \Xi^{\hat{\sigma}} \\
    & \quad + \TORS{\sigma }{\rho}{\mu} \PA{\nu} \Xi^{\hat{\sigma}}- \TORS{\mu}{\sigma}{\nu} \PA{\sigma} \Xi^{\hat{\rho}} \\
    &= 0.
  \end{split}
\end{equation}
This equation yields four different collineation equations,
\begin{equation}
    \Xi^\sigma \pa{\sigma} \Tors{\mu}{*}{*} + \Tors{\sigma}{*}{*} \pa{\mu} \Xi^\sigma - \Tors{\mu}{\sigma}{*} \pa{\sigma} \zeta = 0,
\end{equation}
\begin{equation}
  \Xi^\sigma \pa{\sigma} \Tors{\mu}{\rho}{*} + \Tors{\sigma}{\rho}{*} \pa{\mu} \Xi^\sigma - \Tors{\mu}{\sigma}{*} \pa{\sigma} \Xi^\rho + c \Tors{\mu}{\rho}{*} = 0,
\end{equation}
\begin{equation}
  \begin{split}
    \Xi^\sigma \pa{\sigma} \Tors{\mu}{*}{\nu} + 2 \Tors{\sigma}{*}{[\nu} \pa{\mu]} \Xi^\sigma + 2 \Tors{*}{*}{[\nu} \pa{\mu]} \zeta & \\
    - \Tors{\mu}{\sigma}{\nu} \pa{\sigma} \zeta - c \Tors{\mu}{*}{\nu} &= 0,
  \end{split}
\end{equation}
\begin{equation}
  \begin{split}
    \Xi^\sigma \pa{\sigma} \Tors{\mu}{\rho}{\nu} + 2 \Tors{\sigma}{\rho}{[\nu} \pa{\mu]} \Xi^\sigma + 2 \Tors{*}{\rho}{[\nu} \pa{\mu]} \zeta & \\
    - \Tors{\mu}{\sigma}{\nu} \pa{\sigma} \Xi^\rho &= 0,
  \end{split}  
\end{equation}

%%%%%%%%% BIBLIOGRAPHY %%%%%%%%%
\bibliographystyle{apsrev4-1}
\bibliography{References.bib}

\end{document}

